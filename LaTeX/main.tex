\documentclass[12pt,a4paper,titlepage,headinclude,bibtotoc]{scrartcl}
\usepackage[utf8]{inputenc}
\usepackage{amsmath}
\usepackage{physics}
\usepackage{svg}
\usepackage{float}
\include{julia}



\title{Numerical Fluid Dynamics}
\author{Linus Okko Grimm}
\date{July 2022}

\begin{document}



\maketitle

\begin{gather}
    \pdv{}{t}\va{v} + \left(\va{v}\cdot\nabla \right)\va{v} = - \frac{1}{\rho}\nabla p + \nu\nabla^2\va{v} + g \alpha T \hat{e}_y\label{eq_fst}\\
    \nabla \cdot \va{v} = 0\label{eq_snd}\\
    \pdv{}{t} T + \va{v}\cdot\nabla T = \nabla^2 T + Q\label{eq_thrd}
\end{gather}


\section{Task 1}


%\begin{align}
%    T^\prime &= T\cdot \frac{\kappa}{Q_0L_y^2} \\
%    r ^\prime &= \frac{r}{L_y}\\
%    t ^\prime &= \frac{t\cdot \kappa}{L_y^2}\\
%    v ^\prime &= \frac{r^\prime}{t^\prime} = \frac{r L_y}{ t \kappa} = % v\cdot\frac{L_y}{\kappa}\\
%\end{align}

Using the following transformations
\begin{align*}
    T &=  \frac{T^\prime Q_0 L_y^2}{\kappa}
    &r &= L_y\cdot r^\prime \\
    t &= t^\prime \cdot \frac{L_y^2}{\kappa}
    &v &= v^\prime \cdot \frac{\kappa}{L_y}\\
    \nabla &= \nabla^\prime \frac{1}{L_y}
    &Q &= Q_0\cdot Q^\prime
\end{align*}
% Equation 1
the first equation \eqref{eq_fst}
\begin{equation}
    \pdv{}{t}\va{v} + \left(\va{v}\cdot\nabla \right)\va{v} = - \frac{1}{\rho}\nabla p + \nu\nabla^2\va{v} + g \alpha T \hat{e}_y
\end{equation}
can be written as
\begin{equation}
    \frac{\kappa^2}{L_y^3}\pdv{}{t}\va{v}^\prime + \frac{\kappa^2}{L_y^3}\left(\va{v}^\prime \cdot\nabla^\prime\right)\va{v}^\prime = - \frac{1}{\rho} \nabla^\prime p \frac{1}{L_y} + \nu \nabla^{\prime^2} \frac{\kappa}{L_y^3}\va{v}^\prime + g \alpha T^\prime\frac{Q_0L_y^2}{\kappa} \hat{e}_y.
\end{equation}
With the substitutions $\nu = \text{Pr} \cdot \kappa$, $p = p^\prime \rho \frac{\kappa^2}{L_y^2}$ and $\text{Ra} = \frac{g\alpha Q_0 \kappa}{\nu}\frac{L_y^5}{\kappa^3}$ the common factor $\frac{\kappa^2}{L_y^3}$ can be divided out.
This results in the equation:
\begin{equation}
    \pdv{}{t^\prime}\va{v}^\prime + \left(\va{v}^\prime\cdot\nabla^\prime \right)\va{v}^\prime = - \nabla^{\prime^2} p^\prime + \text{Pr}\nabla^{\prime^2}\va{v}^\prime + \text{Pr}\cdot\text{Ra}\cdot T\hat{e}_y. \label{eq_ad_fst}
\end{equation}
% Equation 2
The same transformation are performed on the second \eqref{eq_snd} and third \eqref{eq_thrd} equation
\begin{equation}
    \nabla \cdot \va{v} = \frac{\kappa}{L_y^2}\nabla^\prime \cdot \va{v}^\prime = 0 \Rightarrow \nabla^\prime \cdot\va{v}^\prime = 0. \label{eq_ad_snd}
\end{equation}
% Equation 3
For the third equation \eqref{eq_thrd} follows
\begin{align}
    \pdv{}{t^\prime} \frac{\kappa}{L_y^2} T^\prime Q_0 \frac{L_y^2}{\kappa} + \va{v}^\prime\cdot \frac{\kappa}{L_y}\nabla^\prime T^\prime Q_0 \frac{L_y^2}{\kappa} &= \frac{\kappa}{L_y^2}\nabla^{\prime^2} T^\prime Q_0 \frac{L_y^2}{\kappa} + Q_0\cdot Q^\prime\\
    \Rightarrow \pdv{}{t^\prime} T^\prime + \va{v}^\prime\cdot\nabla^{\prime}T^\prime &= \nabla^{\prime^2}T^\prime + Q^\prime . \label{eq_ad_thrd}
\end{align}


\section{Task 2}


\section{Task 4}
\begin{align}
    \left(\dv{t} + \abs{\va{k}}^2 \right)\hat{T}_k  + {\left(\reallywidehat{\va{v}\cdot \nabla T} \right)}_k &= \hat{Q}_k \label{eq_spectral_method}\\
     -\abs{\va{k}}^2\hat{T}_k - {\left(\reallywidehat{\va{v}\cdot \nabla T} \right)}_k + \hat{Q}_k &= \dv{}{t}\hat{T}_k \label{eq_spectral_method_rw}
\end{align}
The spectral method \eqref{eq_spectral_method} is used to solve the equation \eqref{eq_ad_thrd}.
The diffusion term $\hat{T}_k\vert \va{k} \vert ^2$ is integrated via Crank-Nicolson \ref{CN}.
Adams-Bashforth \ref{AD} is used for term $\reallywidehat{\va{v}\cdot \nabla}$.
Equation \eqref{eq_spectral_method_rw} can then be rewritten as
\begin{equation}
    \frac{1}{2}\abs{\va{k}}^2 \left( \hat{T}^n_k + \hat{T}^{n+1}_k \right) - \frac{1}{2}\left(3\qty(\reallywidehat{\va{v}^n\cdot \nabla T^n})_k   - \qty(\reallywidehat{\va{v}^{n-1}\cdot \nabla T^{n-1}})_k\right) + \hat{Q}_k= \dv{t}{} \hat{T}_k. \label{eq_methods}
\end{equation}
By discretizing $\dv{}{t} \hat{T}_k$ with 
\begin{equation}
    \frac{\hat{T}^{n+1}_k - \hat{T}^n_k}{\Delta t} = \dv{}{t}\hat{T}_k
\end{equation}
we can rewrite equation \eqref{eq_methods} as
\begin{equation}
    \hat{T}^{n+1}_k = \frac{\hat{T}_k^n + \Delta t\frac{1}{2}\qty(-\abs{\va{k}}^2\hat{T}^n_k - \left(3\qty(\reallywidehat{\va{v}^n\cdot \nabla T^n})_k   - \qty(\reallywidehat{\va{v}^{n-1}\cdot \nabla T^{n-1}})_k\right) + \hat{Q}_k)}{1 + \frac{1}{2}\Delta t\abs{\va{k}}^2}. \label{eq_integrate_T}
\end{equation}
Using equation \eqref{eq_integrate_T} the time integration step is implemented in Julia (Listing \ref{c:intT}).

\section{Task 5}

\begin{figure}[H]
    \centering
    \includesvg[width=0.9\textwidth]{Plots/velocity.svg}
    \caption{$x$ and $y$ components of velocity field $\va{v}$.}
    \label{fig:task5_velocity}
\end{figure}
\colorbox{red}{Vx und Vy sind vertauscht}

From the given velocity field $\va{v}^\star$ (Figure \ref{fig:task5_velocity}) and the temperature field $T^\star$ a heat source distribution $Q^\star$ can be calculated.
Since $\va{v}^\star$ and $T^\star$ are known $Q^\star$ (Figure \ref{fig:task5_q_star}) is given by
\begin{align}
    Q^\star ={}&\va{v}^\star\cdot\nabla T^\star - \nabla^2T^\star\\
    ={}&\begin{pmatrix}\pi\sin\qty(4\pi x)\cos\qty(2 \pi y)\\ -2\pi\cos\qty(4\pi x)\sin\qty(2 \pi y)\end{pmatrix}\cdot\begin{pmatrix}\pdv{}{x}\cos\qty(2\pi x)\sin\qty(2 \pi y)\\\pdv{}{y}\cos\qty(2\pi x)\sin\qty(2 \pi y)\end{pmatrix} - \qty(\pdv[2]{}{x} + \pdv[2]{}{y})T^\star\\
    \begin{split}
        ={}& -2\pi^2\sin\qty(4\pi x)\cos\qty(2\pi y)\cdot \sin\qty(2\pi x)\sin\qty(2\pi y) \\
        &- 4\pi^2\cos\qty(4\pi x)\sin(2\pi y)\cdot\cos\qty(2\pi x)\cos\qty(2\pi y) \\
        &+8\pi^2\cos\qty(2\pi x)\sin\qty(2\pi y) 
    \end{split}
\end{align}
\begin{figure}[H]
    \centering
    \includesvg[width=0.6\textwidth]{Plots/q_star.svg}
    \caption{Temperature source term $Q^\star$.}
    \label{fig:task5_q_star}
\end{figure}
The program from task 4 is now being tested with $Q^\star$, $\va{v}^\star$ and an initial temperature field of $T=0$.
For the Fourier transform a resolution of 32x32 modes was chosen.
After integrating with a $\Delta t = 0.001$ for about 1800 time steps a stable solution was found \mbox{(Figure \ref{fig:task5_stable})}.
\begin{figure}[H]
    \centering
    \includesvg[width=0.6\textwidth]{Plots/task5_stable.svg}
    \caption{Stable solution for $T$ after 1800 time steps with $\Delta t = 0.001$.}
    \label{fig:task5_stable}
\end{figure}
\begin{figure}[H]
    \centering
    \includesvg[width=0.9\textwidth]{Plots/stability.svg}
    \caption{Sum of the absolute differences between temperature fields $T^n$ and $T^{n-1}$.}
    \label{fig:task5_stability}
\end{figure}



\section{Appendix}


\subsection{Crank-Nicolson method}
\label{CN}
\begin{equation}
    x_{i+1} = x_i + \frac{h}{2}\qty[ F\qty(x_i,t_i) + F\qty(x_{i+1},t_{i+1})] 
\end{equation}
\colorbox{red}{QUELLE}
\subsection{Adams-Bashforth method of second order}
\label{AD}
\begin{equation}
        x_{i+1} = x_i + \frac{h}{2}\qty[3F\qty(x_i,t_i) - F\qty(x_{i-1},t_{i-1})] \quad\cite[p.~131]{Ferziger}
\end{equation}

\subsection{Code}
\code{1}{20}{task4.jl}{Time integration of $T$}{c:intT}

\bibliography{quellen}{}
\bibliographystyle{plain}
\end{document}

